\section{Lernziele}
\begin{frame}{Lernziele}{Theorie}
	\begin{itemize}
		\item Unterschied zwischen reinem Map-Reduce und Spark erkennen
		\item Sparks Datenmodell und dessen Implikationen verstehen
		\item Wichtige Bindeglieder und deren Rolle vom Programmieren bis zur Ausführung kennen
		\item Zumindest von der Existenz weiterer Spark-Bibliotheken wissen
	\end{itemize}
\end{frame}

\begin{frame}{Lernziele}{Praxis}
	\begin{columns}
		\begin{column}{0.7\textwidth}
			\begin{itemize}
				\item Verschiedene Aktionen und Transformationen
				\begin{itemize}
					\item kennen,
					\item anwenden und
					\item kombinieren
				\end{itemize}
					können (und das natürlich sinnvoll)
				\talknote<2>{"Wörter Zählen" könnte genau so implementiert werden wie letztes mal}
				\item<2> Wörter-zählen in Spark implementieren können \emoji{wink}
			\end{itemize}
		\end{column}
		\begin{column}{0.2\textwidth}
			\only<2>{
				\begin{tikzpicture}[
					remember picture,
					overlay,
				]
					\owl[yshift=-1.5cm, xshift=1cm, magichat, speech={1,2,3,...}]
				\end{tikzpicture}
			}
		\end{column}
	\end{columns}
\end{frame}